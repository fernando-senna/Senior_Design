6. Safety Requirements

The Eye Tracker will provide the necessary protection to the user while operating the system. Since the systems uses numerous electrical components, the user shall be protected from any electrical shocks, short circuit, or any other electrical issue. In order for the Eye Tracker to operate, it needs an IR LED to track the pupil; this system shall only implement IR LEDs that will not cause any harm to the human eye. Finally, the Eye Tracker will provide the necessary protection to the user in case the electronic components begin to overheat.

3.1 Halogen Light
	3.1.1 Description: The system shall block Halogen Light.
	3.1.2 Source: Dr. McMurrough
	3.1.3 Constraints: Finding the required filter to successfuly block the Halogen Light.
	3.1.4 Standards: None.
	
3.2 IR LED
	3.2.1 Description: The IR LED shall not cause any harm to the user.
	3.2.2 Source: Fernando Do Nascimento
	3.2.3 Constraints: Use a IR LED between 800mm and 900mm.
	3.2.4 Standards: None.
	
3.3 Heat Protection
	3.3.1 Description: The user shall not be harm if the Eye Tracker overheats.
	3.3.2 Source: Fernando Do Nascimento
	3.3.3 Constraints: Designing and building a protective case for the system.
	3.3.4 Standards: None.
	
3.4 Electrical Shock
	3.4.1 Description: The user shall not be harm if the Eye Tracker has electrical problems.
	3.4.2 Source: Fernando Do Nascimento
	3.4.3 Constraints: Providing basic protection and fault protection. Double or reinforced insulation.
	3.4.4 Standards: BS EN 61140
	
3.5 UV Light
	3.5.1 Description: The user shall not be harm by any UV light.
	3.5.2 Source: Dr. McMurrough
	3.5.3 Constraints: Implementing the required filter to block UV light.
	3.5.4 Standards: Enviromental Health & Radiation Safety.
